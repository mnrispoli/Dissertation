%!TEX root = ../dissertation.tex
% the abstract

One of the distinguishing achievements of physics has been the ability to describe the Universe based upon underlying microscopic principles. When there are many participating objects being governed by these microscopic principles, the resultant behavior often harbors emergent collective phenomena that may not be obviously related to the microscopic description. Phase transitions, quantum or classical, are a canonical example of such many-body phenomena that are understood in equilibrium through a framework of thermodynamic constraints, statistical ensembles, and a growth of system-wide correlations at the transition. However, a new class of quantum phase transitions has recently emerged for non-equilibrium systems with behavior that is not captured by this traditional framework. In the case of isolated quantum systems, these transitions occur among excited eigenstates and requires knowledge of the structure of these individual eigenstates in the quantum system. Understanding what drives such a phase transition requires a deeper microscopic knowledge of the role of correlations at the transition. 

The use of ultracold atoms in optical lattices has paved the way for faithfully realizing quantum phase transitions. In particular, the advent of the quantum gas microscope realized an experimental platform that is highly controllable and provides microscopic access to the many-bodied wave functions. All experiments presented in this thesis use such a quantum gas microscope architecture where a quantum degenerate gas of $~^{87}$Rb is used to probe the behavior of non-equilibrium quantum systems in one spatial dimension.

This thesis briefly discusses some studies of entanglement in equilibrium quantum phase transitions near the ground state of the system. In the first experiment, we directly measure the second-order R\'enyi entanglement entropy that develops as a Bose-Hubbard system transitions from a Mott insulator to a superfluid. This is followed by a study of entanglement growth at the transition in the transverse Ising model. 

These studies progress to the investigation of two particular non-equilibrium quantum phases and their transition into one another: quantum thermalization and many-body localization. We investigate the development of entanglement entropy that mimics thermodynamic entropy for an isolated quantum system which locally thermalizes under its own unitary dynamics. We then investigate the only known robust exception to this thermalizing behavior by studying the onset of many-body localization due to disorder. We observe the breakdown of thermalization in this phase and the logarithmically slow growth of entanglement throughout the system, which identifies the localized system to be an interacting one. Lastly, we investigate the role of correlations in the critical dynamics that emerge at high-orders where the system transitions from thermalizing behavior to many-body localization.

%One of the distinguishing achievements of physics has been the ability to describe the Universe based upon underlying microscopic principles. When there are many participating objects being governed by these microscopic principles, the resultant behavior often harbors emergent collective phenomena that may not be obviously related to the microscopic description. Phase transitions, quantum or classical, are a canonical example of such many-body phenomena that are understood in equilibrium through a framework of thermodynamic constraints, statistical ensembles, and a growth of system-wide correlations at the transition \cite{Sachdev2011, Landau1937}. However, a new class of non-equilibrium quantum phase transitions has recently emerged with behavior that is not captured by this traditional framework. In the case of isolated quantum systems, these transitions occur among excited eigenstates and requires knowledge of the structure of these individual eigenstates in the quantum system\cite{Nandkishore2015, Zvyagin2016, Tauber2017, Alet2018, Heyl2018}. Understanding what drives such a phase transition requires a deeper microscopic knowledge of the role of correlations at the transition. 
%
%%In this thesis, I present experiments that realize both equilibrium and non-equilibrium quantum phase transitions with the ability to probe the development of correlations in the system.
%%Understanding such many-body phenomena often relies on effective macroscopic models to describe such behaviors. 
%The use of ultracold atoms in optical lattices has paved the way for faithfully realizing quantum phase transitions\cite{Greiner2002, Bloch2008, Bloch2012, Bakr2010, Simon2011, Lewenstein2012}. In particular, the advent of the quantum gas microscope realized an experimental platform that is highly controllable and provides microscopic access to the many-bodied wave functions \cite{Gillen2009, Bakr2009}. All experiments presented in this thesis use such a quantum gas microscope architecture where a quantum degenerate gas of $~^{87}$Rb is used to probe the behavior of non-equilibrium quantum systems in one spatial dimension.
%
%This thesis briefly discusses some studies of entanglement in equilibrium quantum phase transitions near the ground state of the system. In the first experiment, we directly measure the second-order R\'enyi entanglement entropy that develops as a Bose-Hubbard system transitions from a Mott insulator to a superfluid \cite{Islam2015}. This is followed by a study of entanglement growth at the transition in the transverse Ising model. 
%
%These studies progress to the investigation of two particular non-equilibrium quantum phases and their transition into one another: quantum thermalization and many-body localization. We investigate the development of entanglement entropy that mimics thermodynamic entropy for an isolated quantum system which locally thermalizes under its own unitary dynamics\cite{Kaufman2016}. We then investigate the only known robust exception to this thermalizing behavior by studying the onset of many-body localization due to disorder. We observe the breakdown of thermalization in this phase and the logarithmically slow growth of entanglement throughout the system, which identifies the localized system to be an interacting one\cite{Lukin2019}. Lastly, we investigate the role of correlations in the critical dynamics that emerge at high-orders where the system transitions from thermalizing behavior to many-body localization\cite{Rispoli2018}.
 
%
%Due to the tunability afforded by our quantum gas microscope, we are able to study the critical dynamics of this non-equilibrium transition and microscopically resolve how the constituents in the system correlate in their dynamics. near criticality. 
%
%Theoretically understanding the exact structure of the eigenstates at such a phase transition requires numerically intractable results. This necessitates the physical realization of such systems and motivates the approach of a microscopic study that illuminates the underlying complex structure of the eigenstates at the transition. The presented experiments in this thesis demonstrate the ability to derive instructive results from continued experiments in the understanding the fundamental properties of correlations in such eigenstate phase transitions. 
%
%Such phase transitions realize a new class of phase transitions that fundamentally modify a traditional statistical mechanics approach where the only viable approach is a microcanonical model that includes a single eigenstate and precludes the concept of an ensemble. 