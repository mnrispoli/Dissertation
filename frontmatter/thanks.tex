%!TEX root = ../dissertation.tex
% the acknowledgments section

%\begin{savequote}[75mm]
%Are we still having fun?
%\qauthor{Dr. Daniel Greif}
%\end{savequote}
\begin{minipage}[t]{75mm}
 \begin{quotation}
 \emph{Are we still having fun?}
\qauthor{Dr. Daniel Greif}
 \end{quotation}
 \end{minipage}
 
 \vspace{10mm}

\newthought{The physics we study occurs in a vacuum, but our scientific success does not}\footnote{Thank you Christie for coining this very appropriate phrase.} -- There are an enormous number of people that are either directly or indirectly responsible for making all the results presented in this thesis possible.

I would like to thank my advisor, Markus Greiner, for six years of guidance and accepting me into his group considering that I arrived with zero AMO experience. You are an an endless supply of intuition and experimental voodoo that never ceases to amaze me.

Of course, my experience working in the Rubidium Lab has been predominantly related to my overlap with a group of very talented lab mates. I would like to thank them for not only keeping work in the lab fun, but also for teaching me a number of lessons over the years. Alex Ma taught me that everything will be ``okay". Philipp Preiss patiently taught me how to run the experiment and even walked me through simple calculations that were certainly only useful for one of us (me). Not to mention he has continued to answer emails about the experiment years after having left and is always a pleasure to hang out with whenever conferences bring us back together. Eric Tai taught me how to create -- his attention to detail and care in fabrication is something I will always aspire to but probably never obtain. Thank you for your patience in listening to my ideas and work through problems with me. Not to mention all the time spent outside of the lab: going to the gym, cycling the Minuteman Bikeway, advice on getting a job, and going on adventures in Taiwan and Singapore. Alex Lukin taught me persistence and how to ask the correct conceptual questions. I have worked with no one else as closely in lab and I have no doubts that the latest experiments would have never succeeded without his help and guidance. He has independently provided an enormous amount of joy and entertainment both in and out of the lab -- I think everyone would agree that life without him would have been significantly less memorable or fun.

Then there are the new-\emph{ish} lab members who deserve significant credit and thanks too -- they have already taught me a great deal and I am excited to see where their enthusiasm and talent will lead the experiment. Robert Schittko re-taught me English, how to actually makes figures beautiful, and to never stop asking questions. Sooshin Kim showed me what is \emph{actually} numerically possible. The critical behavior paper would have certainly been less successful without him. Joyce Kwan has taught me how to be deliberate and persistent in conversations: both when looking for a job and with faculty members during grad school. Her diligence and optimism provided a significant improvement to the lab.
% Tim Menke, for not only working diligently through the initial gauge-field lattice days and keeping in touch over the years while you go off to do new and interesting physics.

The Rubidium Lab has also been fortunate enough to have had a great number of talented postdocs that have led the experiment along a successful scientific path and kept the group harmonious and enjoyable to work in. Rajibul realized the first steps towards the subsequent measurements of entanglement in an isolated quantum system. I'm additionally thankful for his willingness to work with me even during my first days in the lab. Adam's patience and willingness to talk about all things physical, political, pet related or otherwise not only significantly improved the quality of research in the lab, but the general work environment. He kept the lab organized and was a wonderful mentor. Julian brought in a great deal of experimental expertise and in-depth knowledge of equilibrium phase transitions. His guidance brought the MBL critical behavior paper to fruition and without him the lab would have been a disorganized mess.

Of course, the Greiner Group is composed of more than just the Rubidium Lab. My colleagues from the Lithium and Erbium labs have helped develop me both professionally and as an individual over the past six years. Not to mention that they have provided an endless amount of entertainment and fun through lunch conversations, group hikes, group skiing, and conferences. Thank you to the Lithium lab members over the years: Sebastian Blatt, Florian Huber, Max Parsons, Daniel Greif, Anton Mazurenko, Christie Chiu, Geoffrey Ji, Muqing Xu, and Justus Br\"uggenj\"urgen. Thank you to the Erbium lab members over the years: Susannah Dickerson, Greg Phelps, Anne H\'ebert, Aaron Krahn, Furkan \"Ozt\"urk, Sepher Ebadi, and Emily Tiberi.

Even before coming to the Greiner Group, a number of professors and teachers influenced my educational path and academic career that brought me this PhD program: Stephen Sekula, Jodi Cooley, Randall Scalise, Nathan Huntoon, Janet Schofield, and Scott Wiley -- just to name a few.

Additional acknowledgement and thanks also goes out to the administrators, staff, and technical experts in the department: Lisa Cacciabaudo, Carol Davis, Stuart McNeil, Stan Cotreau, Jim MacArthur, Silke Exner, Manny \& Danny, Samantha Dakoulas, and Clare Ploucha. Without them, not only would much of the group not successfully run, but the entire department would probably fall apart. Jim MacArthur deserves a separate and special acknowledgement for his willingness to talk to me about all things that I brought up to him throughout the years -- electronic and otherwise. He \emph{always} set aside whatever he was doing and would spend time to discuss the topic with me. He additionally acted like a second advisor and helped me a great deal with figuring out what to do with my life after graduation.

The past six years have also been more than just research. I have been fortunate enough to have been supported and kept sane by all the good times from old friends. Some that go back as far as high school:  Gabriel, Tony, Nick, Allen. Another group of friends, whom I was lucky enough to meet a bit later in life during undergrad: Cristina, Ali, Brian, Dan, Tori, Keith, and Isaac. Of course, the most recent and influential group of friends on my personal development during my PhD were found while here at Harvard: all of my entering class, my lab mates, and other students in the physics department either at Harvard or MIT that I've had the pleasure of meeting along the way: Ahmed, Annabelle, Fabian, Alp, Olivia, Ruffin, Ivan, Tim, Victor, and Lee.

Some of these friends deserve special individual acknowledgements for their significant contributions to my PhD experience:

Thank you Anne for being a constant source of joy, sanity, and support over the past years: eating raclette, joining me on cycling adventures, eating waffles, taking care of my cat, reading this thesis, and being a great desk neighbor -- you're a wonderful friend and the entire Greiner Group is lucky to have you in it.

Thank you Christie, as my fellow previously-non-AMO colleague entering the same year into the Greiner Group, you have been a wonderful lab mate that has always been willing to share your thoughts and support in both professional and personal matters over the years. 

Thank you Julian for being such an inclusive postdoc and good friend. You not only supported me in professional matters in the lab but even welcomed me into your home when at conference in Germany. I am additionally very grateful for all the times you have selflessly offered to take care of my cat when I've been away. 

Thank you Dom for being my first friend in grad school and being a great roommate over the years. Thank you for helping me through difficult times in my PhD: both work and non-work related problems. Thank you also for being the genesis of many great memories -- including one where we opened a bottle of wine with a shoe.

Thank you Louis, Jae, Michael, and Ron for being a great friend group from my entering class in grad school. Louis is a fellow CUA member and AMO experimentalist with far more technical knowledge than I will ever have. You and Rachel have been a wonderful couple to hang out with -- whether it was eating paella in Barcelona or having nearly-unrememborable experiences at beerfest here in Boston. Jae is the first person I met in grad school and was one of my first year officemates -- you've not only been a great friend over the years, but your openness made me feel welcomed on the first day of the program. Michael was another officemate from my first year who introduced me to the importance of Cinnabon and the vastness of the internet -- your friendship and empathy throughout the years has been invaluable and I hope you find your way back to the east coast in the years to come. Ron is another AMO physicist whose pragmatic down-to-earth-edness has always served as an example of how to live a less stressful life -- separately, an example of how to bake amazing chocolate chip cookies.

Thank you Adam and Annalise (and Maddy), while I had the good fortune of working with Adam in lab, I had the even greater fortune of befriending all three of you! Adam is responsible for teaching me a great deal about physics and has kept an eye out for me both in and out of lab. Thanks to Adam, I was able to meet Annalise who, combined with her family, make up the sweetest group of people I have ever met -- you're an inspiration and example of the goodness possible in people. Maddy is, of course, amazing.

Thank you Daniel and Renate for being some of the closest friends I've had the pleasure of meeting in grad school. While Daniel and Renate are both AMO physicists, I have never had the pleasure of working together directly with them. However, they have always been willing to listen to me incessantly bug them about physics questions (especially Daniel about topology). Thank you for always reminding me that physics should be fun. Outside of the lab, the countless late nights of board games, waffle-seeking bike rides, and raclette dinners have made up a great deal of my fond memories for the past six years.

Thank you Anton and Caitlin (and Chloe) for being close friends in grad school. Anton deserves a separate acknowledgement for helping me land my job coming out of grad school -- something I am extremely grateful for. Anton is the embodiment of selfless giving and was willing to help me with whatever often thankless lab-related task arose. Of course, this selflessness isn't limited to lab: he has helped me move furniture more than once and listened to me complain and ask advice for all things job related. I'll cherish many memories from grad school together (even the one where we opened a wine bottle with a fillet knife) and look forward to even more in New York. Thanks to Anton, I also got to meet his better thirds (yes it is a quasi-polyamorous relationship): Caitlin and Chloe. Caitlin is, perhaps unsurprisingly as Anton's partner,  another incredibly sweet individual that I look forward to now having the opportunity to spend more time with given my new job. And, of course Chloe, as the most opinionated cat I have ever met. She is responsible for some of the best cat stories that I know and, somehow, most of the cat photos in my phone are of her.

Pepper, as my own cat, of course deserves credit for providing a great deal of comfort and entertainment over the years. She has greeted me at the door almost every day and has spent the most time going over this thesis with me -- either by sitting on me, the laptop, or the printed document itself and staring, perhaps carefully, into it. 

My family have supported me with their unconditional love and support throughout my life. My parents Valda and Frank have always encouraged me to follow what I find interesting and provided me with every opportunity I could have ever hoped for -- this would have not been possible without you. My brother Stephen and my sister-in-law Jeanine have supported me throughout my academic career and I am proud to finally be finishing a graduate degree in their footsteps.

There is a final, very special acknowledgement that goes to my partner Melis Tekant. The most important result of my entire PhD experience is that it introduced us! You have been my ceaseless supporter I don't believe I can ever truly deserve. A PhD can be a trying experience and it is thanks to your love, help, and good judgement that I have managed to successfully finish this degree at all. You are a loving companion that has picked me up when I've been down and taken me on a number of adventures that I will treasure forever. Through you, I have even gained a second family that has welcomed with open arms and shown me more warmth than I would have ever dreamed possible. I can't wait to see what the future has in store for us and look forward to every new experience we'll share together. Seni \c{c}ok seviyorum.