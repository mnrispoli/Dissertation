%!TEX root = ../dissertation.tex

%\begin{savequote}[75mm]
%Simulation is a valid way of answering any question.\footnotemark
%\qauthor{Stephen Sekula}
%\end{savequote}



\chapter{Introduction}
\label{introduction}

%\footnotetext{\footnotemark a way to add footnotes to quotes}

Punchy very general intro about quantum simulation, correlated systems, and intractable problems

We have developed a wildly successfful microscopic theory of the universe via quantum mechanics (no, it doesn't seem to quite capture everything, but it's remarkably good) and we also have a very comfortable intuitive set of macroscopic phenomena we all know, love, and experience on a daily basis. Can these things become compatable? 

Emphasize the difference between larger, more particles and the every day world which is an open-system.

The key here maybe be in how information and correlations develop, something that requires an intense amount of tunability and versatility in an experimental apparatus. In walks the QGM where we have access to an unprecedented amount of information and can, for example, capture all spatial correlations in a wave-function.




